\section{This thesis}

This thesis titled \say{Rotational and vibrational action spectroscopic studies on cold molecular ions} discusses action spectroscopic techniques employed to characterise molecular ions in a cryogenic ion trap spectroscopically. Molecular ions relevant to astrochemistry, especially in the interstellar medium and planetary atmospheres, are mainly focused on and are discussed in respective chapters.\\

\textbf{Chapter \ref{chapter:methods}} \emph{\say{Experimental and theoretical methods}}:  gives an overview of the experimental setup used in this study, including the ion source, cryogenic trap and detector. A detailed description of the action spectroscopic methods employed in this thesis to characterize molecular ions' rotational and vibrational transitions is discussed. Technical details such as determining number density with uncertainty, calibration of instruments and instrument setup are discussed in detail.\\

\textbf{Chapter \ref{chapter:C3H3+}} \emph{\say{Laboratory gas-phase vibrational spectra of \texorpdfstring{[C$_3$H$_3$]$^+ $}\ isomers and isotopologues by IRPD spectroscopy}}: In this chapter, we investigated broadband gas-phase Ne-IRPD spectra of both linear and cyclic forms of [C$_3$H$_3$]$^+$ and reported the first gas-phase IR spectra of the corresponding [C$_3$D$_3$]$^+$ isomers. Various high-level coupled-cluster methods are benchmarked. WE also investigated the isomeric ratio quantification of [C$_3$D$_3$]$^+$ produced with different ion source conditions and precursors.\\

\textbf{Chapter \ref{chapter:CH3CNH+}} \emph{\say{Infrared predissociation  spectroscopy of protonated methyl cyanide,  \texorpdfstring{CH$_3$CNH$^+$ }\ }}: In this chapter we present a comprehensive experimental and quantum-chemical study of the vibrational spectrum of Ne-CH$_3$CNH$^+$. A focus is on the influence of the weakly-bound neon atom on the infrared pre-dissociation experiments. We also demonstrated an efficient computational approach to provide accurate estimates of anharmonic 
 vibrational frequencies of the bare ion and complex.\\

\textbf{Chapter \ref{chapter:HC3N+}} \emph{\say{A vibrational action spectroscopic study of the Renner-Teller and spin-orbit affected cyanoacetylene radical cation \texorpdfstring{HC$_3$N$^+$}\ }}: In this chapter, we present the investigation of the vibrational transitions of HC$_3$N$^+$, an open shell linear molecular ion. The breakdown of the Born-Oppenheimer approximation due to the Renner-Teller (RT) effect (vibrational coupling) is analysed using an effective Hamiltonian approach. The influence of the tag in IRPD, especially on the bending modes of RT-affected open-shell, is discussed in detail.

In these first chapters, the vibrations transitions on the potential candidates of interstellar molecular ions are experimentally and theoretically investigated in detail, and discussed from various perspectives, such as isomer quantification, the influence of tag on smaller molecular ions and RT-affected open-shell species. In the following chapters, the investigation focuses on high-resolution pure-rotational action spectroscopy. Rotational transitions provide distinct molecular fingerprints, and importantly, due to their low excitation temperature, most interstellar species are identified through their rotational transitions. The molecular ions discussed above are potential interstellar and also (exo-)planetary candidates. On the basis of their vibrational characterisation, rotational characterisation will be followed in future. Therefore, the following two chapters discuss the implementation and investigation of a novel rotational action spectroscopic technique (ROSAA), which utilises a change in rare-gas atoms attachment rates for measuring pure rotational transitions of bare ions. The method is illustrated for a closed and for the first time, an open-shell molecular ion. \\

\textbf{Chapter \ref{chapter:CD+}} \emph{\say{Kinetics of CD\texorpdfstring{$^+$ }\ ion with He buffer gas}}: In this chapter, we report a systematic study and detailed analysis of the CD$^+$ reaction with helium buffer gas atoms with and without the presence of radiation resonant with the $J=0-1$ rotational transition of CD$^+$ ion. This is important in investigating the ROSAA signal intensity (i.e., measured rotational transition intensity) processes using numerical simulation. Consequently, a robust numerical simulation model was developed to predict the intensity of the rotation transition of the molecular ions of interest.

\textbf{Chapter \ref{chapter:CO+}} \emph{\say{The Zeeman effect in CO\texorpdfstring{$^+$ }\ observed with rotational action spectroscopy}}: In this chapter, the high-resolution rotational transition of CO$^+$, an open-shell molecular ionic species, is investigated using the ROSAA technique. In addition to an unpaired electron fine structure splitting, a (partly) resolved hyperfine Zeeman splitting is observed due to Earth's magnetic field. The measured signal intensity is investigated using the developed numerical simulation model.\\
