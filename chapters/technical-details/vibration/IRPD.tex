\subsection{Experimental method}
\label{subsec:IRPD}

The action spectroscopic method employed to record the vibrational transitions of the molecular ions in this study is the well-known \qt{infrared predissociation (IRPD)} spectroscopy, as introduced in Section \ref{subsec:action:methods:vibrational:IRPD}.

In our IRPD experiment, the primary target ions are formed in the ion source and mass filtered into the 22-pole ion trap. A short He or He:Ne mixture pulse (50-100 ms) is introduced into the trap (typically at 5-9 K range). The molecular ion can form a weakly bound complex at low temperatures ($< 10$ K) and high number density ($10^{15}$ \percc ) via three-body collisions, as discussed in more detail in Section \ref{subsec:CD+-kinetics}. The tagging efficiency in forming the complex depends on the interaction strength between the molecular ion and the tag (He or Ne). This study typically obtains $> 10 \%$ tagging yield; more details on different molecular ions are described in their respective chapters. For tagging partners, the neon atom is often preferred, although the helium is better suited for IRPD studies due to its lower polarizability, lower binding energy, and thus  minimal perturbation on the ionic vibrational frequencies. However, for the same reason, since the He-complex is very weakly bound, it dissociates due to trap heating from a high-power radiation source, such as the free-electron laser employed in this study. The influence of the Ne tag on bare ion vibrational modes is discussed in corresponding chapters.

The trap contents are stored in the ion trap for about 1-2 s while being irradiated with pulsed IR radiation as a function of frequency, and the formed complexes are then mass filtered and counted. The dissociation of the complex at resonance transition yields the signal in the form of a depletion. The IR radiation source and data normalising procedures are described in the following sections.
