\subsection{Experimental method}
\label{subsec:ROSAA}

The ROSAA action spectroscopic technique is employed in this work to record pure rotational transitions. In this section, we shall discuss this technique in detail.

Initially, the primary target molecular ions are produced from an ion source (see Section \ref{subsec:setup:ion-source}) by electron ionization from a neutral precursor using either storage or non-storage ion sources. A short pulse of the isolated molecular ion of interest is injected into the trap and stored for a specified time, typically $\sim 600$ ms for rotational action spectroscopic experiments, with continuous inflow of either pure He or He:Ne mixture buffer gas for collisional cooling and complex formation. At low temperatures (around 5-6 K and 6-15 K trap ambient temperature for helium and neon tag, respectively) and high number density $\sim10^{14}$ \percc\ of gas inflow, the buffer gas atoms will readily attach to the target ion by ternary association and can dissociate by collision-induced dissociation.

The ROSAA technique utilises the change in the rare gas atoms' attachment rate to a molecular ion (M$^+$) depending on its internal excitation, i.e., ions with a specific rotational quantum number $J$ have different attachment rate coefficients for forming HeM$^+$ clusters with Helium (see Section \ref{subsec:ROSAA-simulation}). The ternary association and collision-induced dissociation rate coefficients can be experimentally measured by following corresponding reactants and products ion counts as a function of trap time. These rate coefficients are a weighted averaged rate coefficient over the thermal population of rotational levels, i.e., the Boltzmann distribution close to the nominal trap temperature, reached by He collisional excitation rates of the order of $\sim 10^4$~s$^{-1}$ at the typical number densities ($\sim 10^{14}$ \percc) used in these experiments. Upon resonant excitation, the thermal equilibrium distribution is disturbed by competing radiative processes (typically with comparable rates of $\sim 10^5$~s$^{-1}$ for the M$^+$ rotational transitions), leading to a change in the attachment rate and thus the number of formed complexes. Hence, the measured signal intensity ($S$) is given as the observed change (in \%) of the number of He-M$^+$ complexes formed between the set frequency ($I_{ON}$) and a fixed reference frequency ($I_{OFF}$, offset about 3 $\sim$ GHz from scanning range), and scaled by $I_{OFF}$, i.e., $ S=(I_{OFF} - I_{ON})/I_{OFF} $, after storing for a fixed time of typically $\sim$ 600 ms in the trap at each data point (see Sections \ref{subsec:rot:radiation-source} and \ref{subsec:rot:power} for radiation source details). The spectra are measured in typically $10\sim$ kHz steps.
