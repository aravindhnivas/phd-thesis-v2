\subsection{Quantum chemical calculations}
\label{sec:QC-calculations}
The molecular vibration and rotation laboratory spectroscopic studies reported in this work are supported by quantum chemical calculations as described in Section \ref{sec:mol-vibration} and \ref{sec:mol-rotation}. This section briefly discusses the methods and programs used to employ quantum chemical calculations.

Initially, a potential energy surface is computed to characterise the molecular ion of interest, and energetically stable structures are derived and structurally optimised. These investigations are performed at the coupled-cluster singles and doubles (CCSD) level augmented by a perturbative treatment of triple excitations, CCSD(T) \cite{raghavachari_fifth-order_1989}, in combination with atomic natural orbital (ANO0, ANO1, and ANO2) basis sets from Alml\"of and Taylor \cite{almlof_general_1987, almlof_atomic_1991} as well as the correlation-consistent valence basis set cc-pVDZ \cite{dunning_gaussian_1989} in the frozen core (fc) approximation. The equilibrium geometries have been calculated using analytic gradient techniques  \cite{watts_open-shell_1992}.

Vibrational modes of molecular ions of interest are further investigated by computing harmonic frequencies using numerical differentiation of gradients \cite{lee_analytic_1991, watts_coupledcluster_1993}. Second-order vibrational perturbation theory (VPT2) \cite{mills_32_1972} has been employed for anharmonic calculations. All CCSD(T) calculations have been carried out using the CFOUR program package  \cite{matthews_coupled-cluster_2020,harding_parallel_2008}.

The influence of the rare gas tag on the IRPD spectrum is further investigated by computing interaction energies (using either CFOUR \cite{matthews_coupled-cluster_2020} or PSI4 \cite{smith_psi4_2020} program) as well as harmonic frequencies of the ionic complexes. For the complexes, the Basis Set Superposition Errors (BSSE) \cite{liu_accurate_1973} are addressed using i) the counter-poise (CP) method introduced by Boys and Bernardi \cite{boys_calculation_1970}, i.e. by calculating CP-corrected CCSD(T) interaction energies at each geometry, and ii) higher-order symmetry-adapted perturbation theory, SAPT2+3 \cite{jeziorski_perturbation_1994, hohenstein_density_2010}. 

Measured pure rotational and vibrational transitions (see Chapter \ref{chapter:CO+}) are fitted with an effective Hamiltonian approach using the Pgopher program \cite{western_pgopher_2017} to derive molecular spectroscopic constants. Further computational details specific to certain ionic species are reported in their respective chapters in detail. The next section focuses on experimental technical details such as determining number density.
