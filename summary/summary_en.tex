\chapter{Summary and outlook}
Interstellar molecules were once thought to be impossible to exist in space due to the extreme conditions of the interstellar medium (ISM). The extremely low densities and temperatures, as well as the intense radiation and high-energy particles present in space, were believed to prevent the formation and survival of molecules.

However, this understanding began to change in the early 20th century with the discovery of CH, CN and CH$^+$ molecular species in the ISM based on four sharp absorption lines seen in the optical spectra of several distant stars using Mount Wilson Observatory. The later discovery of other molecules, such as hydroxyl radical, water, ammonia and formaldehyde in the ISM by radio astronomy in the 1960s provided evidence for the existence of a wide variety of molecular species in space.

This revelation was significant because it challenged the prevailing view of the ISM and led to the development of new theoretical models about the chemical processes that occur in space. One such revelation is that ion-neutral reactions are a crucial part of the chemistry in the interstellar medium, driving the formation of complex molecules and influencing the chemistry of the ISM. This also provides new insights into the origins of life in the universe, as the discovery of complex organic molecules in space suggests that these building blocks of life may be prevalent throughout the cosmos.

Spectroscopic methods are crucial in studying interstellar molecules and (exo-)planetary atmospheres as well. These methods have been and will continue to be, crucial in discovering new interstellar molecules and understanding the chemistry of space and (exo-)planetary atmospheres. The introduction chapter \ref{chapter:intro} gives a much more detailed introduction to molecular ions in space, astrochemistry and the development of spectroscopic methods for studying molecular ions in a cryogenic ion trap.

Cryogenic ion traps are important tools in the field of spectroscopy, particularly in the study of molecular ions. They allow us to perform gas-phase action spectroscopy on cold molecular ions at $< 10$ K ambient temperature, i.e., simulating ISM conditions. The combination of action spectroscopic methods with cryogenic ion traps provides several advantages over traditional spectroscopy methods, such as mass selection and storage in a cold ion trap, uncontaminated spectra and narrow line widths. Chapter \ref{chapter:methods} gives a detailed account of the instrumental setup and various measurement techniques used in this study for the rotational and vibrational transition characterisation of cold molecular ions.

The following chapters \ref{chapter:C3H3+}, \ref{chapter:CH3CNH+} and \ref{chapter:HC3N+} provide a characterisation of the vibrational transitions of potential candidates of interstellar molecular ions. They are experimentally (using IRPD) and theoretically investigated in detail, which is briefly summarised below:

The C$_3$H$_3^+$ hydrocarbon molecular ion is an important intermediate in various astrochemical environments such as the interstellar medium, cometary comae, and planetary atmospheres. Therefore, in chapter \ref{chapter:C3H3+} we characterise [C$_3$H$_3$]$^+$ and [C$_3$D$_3$]$^+$ isomers. Since these ions have two stable isomers, the cyclic cyclopropenyl cation, c-C$_3$H$_3^+$, and the linear propargyl cation, H$_2$C$_3$H$^+$, isomer quantification with different ion-source conditions and precursors are investigated using kinetic scans, i.e., monitoring target ion-complex depletion as a function of time in the presence of resonant radiation on a chosen vibrational transition specific to one isomeric species.

Methyl cyanide (CH$_3$CN) was among the first polyatomic molecules detected during radio-astronomical observations of the ISM. As methyl cyanide has a high proton affinity, much larger than that of H$_2$, its protonated version (CH$_3$CNH$^+$) can form effectively via exothermic proton transfer from H$_3^+$ to CH$_3$CN in the interstellar medium. Chapter \ref{chapter:CH3CNH+} focuses on CH$_3$CNH$^+$ providing a comprehensive experimental and quantum-chemical study of the vibrational spectrum of Ne-CH$_3$CNH$^+$. The influence of the weakly-bound neon atom on the IRPD experiments is investigated in detail using various computational methods.

Cyanopolyynes are a class of molecules consisting of a chain of carbon and nitrogen atoms. The simplest cyanopolyyne is cyanoacetylene (HC$_3$N), one of the ISM's most widespread polyatomic species. It has been observed in various astronomical environments, including the interstellar medium, comets, and the atmosphere of Titan, one of Saturn's moons. However, its highly reactive cationic counterpart (HC$_3$N$^+$), which is postulated to form from the ionisation of HC$_3$N by cosmic rays or UV photons, has yet to be detected in space. Therefore, in chapter \ref{chapter:HC3N+} we investigate HC$_3$N$^+$ through vibrational studies. Interestingly, the HC$_3$N$^+$ molecular ion is an open-shell linear species which results in pronounced Renner-Teller (RT) effects, i.e., vibronic coupling. Therefore, the breakdown of the Born-Oppenheimer approximation due to the RT effect is analysed using an effective Hamiltonian approach. The influence of the tag in IRPD, especially on the RT-affected bending modes, is discussed in detail.

The last two chapters, i.e., \ref{chapter:CD+} and \ref{chapter:CO+} focus on the investigation of high resolution pure-rotational action spectroscopy. Rotational transitions provide similar or even more distinct molecular fingerprints than vibrational transitions. Due to their low excitation temperature, most interstellar species are identified through their rotational transitions. The rotational high-resolution rotational action spectroscopy technique employed in this thesis is called ROtational State-dependent Attachment of rare gas Atoms (ROSAA), which utilises a change in rare-gas atom attachment rates for measuring pure rotational transitions of bare ions. The main motivation of the last two chapters is to understand the ROSAA process and its signal intensities in detail with the support of numerical simulation results. Therefore, implementation and investigation of this novel rotational action spectroscopic technique are illustrated for a closed-shell (CD$^+$ in chapter \ref{chapter:CD+}) and, for the first time, an open-shell molecular ion (CO$^+$ in chapter \ref{chapter:CO+}). Chapter \ref{chapter:CD+} experimentally investigates the kinetics of the ion-neutral three-body collisional process, subsequently systematically developing a ROSAA kinetic model, while chapter \ref{chapter:CO+} gives a detailed account of utilising this model for the open-shell species CO$^+$ which was rotationally characterised with resolved Zeeman splitting due to the Earth's magnetic field.\\


In the following paragraph, a brief outlook is provided for continuing further in future studies.

As discussed above, measuring the rotational transitions of molecular ions can increase the possibility of their astronomical detection significantly. Therefore, the molecular ions which were vibrationally characterised in this thesis shall be investigated further using the ROSAA method. The vibrational spectra and quantum-chemical calculations provided in this work form, together with the developed numerical model, a solid basis for these future studies. For example, the propargyl cation $l-$C$_3$H$_3\ ^+$ (i.e., CH$_2$CCH$^+$) is an important candidate to study since its neutral form had been recently (2021) detected in the TMC-1 region and was shown to be one of the most abundant radicals to have ever found. Likewise, it would be interesting to study rotational transitions of the singly and doubly deuterated forms of $c-$C$_3$H$_3\ ^+$ (the undeuterated cyclic variant $c-$C$_3$H$_3\ ^+$ does not possess permanent dipole moment).

The main advantages of the ROSAA technique are that (1) the possibility of helium attachment to any cation at 
low temperature ($< 15$ K) makes it to study very challenging highly reactive molecular ions, 
and (2) the technique is very sensitive compared to traditional absorption methods, 
i.e., it only requires a few hundred ions to record the spectroscopic signal.
However, for nosiy signal of ion counts, one would require to spend a longer integration time
to obtain a high signal-to-noise ratio. For instance, it takes around $\sim 28$ hrs, 
for scanning 50 MHz in 10 KHz steps for 10 iteration cycles. This is a vital step, because 
the ROSAA spectroscopic signal intensity is directly related to the change in the number of ion-complex 
formed as a result of change in internal rotational level population distribution at resonant frequency.
The weaker signal ara also expected when there are only very less ions 
resides in a target rotational level at a given temperature, hence a very small change 
in population can be induced at resonance. However, it is indeed possible to detect 
such small changes by longer integration time as discussed before, even for ions which are very difficult 
to produce and challenging to study.

Therefore, in future, one can try to utilise the ROSAA technique to 
study and provide accurate rotational constants, for example, carbocations and 
nitrogen bearing carbon-chain radical ions, for their astronomical 
identifications. Since they are an important intermediate for many organic molecules 
which are observed abundantly in ISM.

% The CH$_2^+$ molecular ion is one such  well-known potential interstellar candidate. Therefore, a study focusing on a CH$_2^+$  and its deuterated isotopologues is very interesting and vital. The CH$_2^+$ ion also possesses interesting intramolecular dynamics such as Renner-Teller interactions and quasi-linear structure. Several detailed studies both experimentally and theoretically had been carried out on CH$_2^+$ molecular ions because of the aforementioned reasons. However, experimental gas-phase pure high-resolution rotation data in the vibronic ground state is not available. Therefore, in future, one can try to utilise the ROSAA technique to study and provide accurate rotational constants for its astronomical detection.

In another scenario, we have tried to measure the rotational transitions for linear isomer of [CH$_2$CN]$^+$ ion, i.e., H$_2$CCN$^+$ (cyanomethyl) molecular ion, which have recently studied in our group using IRPD spectroscopy. But have failed to detect any ROSAA signal. This should be further investigated using the developed ROSAA model. Therefore, in future studies, the model reported in this thesis can be readily utilised to first understand the expected signal intensity at a given temperature, number density and power, and decide the possibility of applying the ROSAA technique. The search for the rotational transitions was based on high-level quantum-mechanical calculations, which should limit the search range to within 1 GHz, for which the previously recorded IRPD data provided good benchmarking.

% On the other hand, for vibrational characterisation, an interesting study proposal had been submitted for a FELIX beam time, which is yet to be performed is \emph{Spectroscopic and structural characterization of the small carbon clusters C$_4^+$ and C$_7^+$: Transition from linear to cyclic structures ?}. Using the methods developed in our group as reported in this thesis extensively, future studies to disentangle and quantify isomeric mixtures of these smaller carbon cluster cations using IRPD are vital due to their potential importance in circumstellar and interstellar carbon chemistry.
