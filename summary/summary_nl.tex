\chapter{Samenvatting en vooruitzichten}

{\selectlanguage{dutch}
Interstellaire moleculen werden ooit als zeer onwaarschijnlijk beschouwd om te bestaan in de ruimte vanwege de extreme omstandigheden van het interstellaire medium (ISM). De extreem lage dichtheden en temperaturen, evenals de intense straling en hoogenergetische deeltjes die aanwezig zijn in de ruimte zouden de vorming en overleving van moleculen voorkomen. 

Dit begon echter te veranderen aan het begin van de 20$^e$ eeuw met de ontdekking van de moleculen CH, CN en CH$^+$ in het ISM op basis van vier scherpe absorptielijnen die werden gezien in de optische spectra van verschillende verre sterren met behulp van het Mount Wilson Observatorium. De latere ontdekking van andere moleculen in het ISM door middel van radioastronomie in de jaren 60, zoals het hydroxylradicaal, water, ammoniak en formaldehyde, leverde bewijs voor het bestaan van een breed scala aan moleculaire soorten in de ruimte. 

Deze ontdekking was belangrijk omdat het de heersende opvatting over het ISM betwistte en leidde tot de ontwikkeling van nieuwe theoretische modellen over de chemische processen in de ruimte. Een voorbeeld van een ontdekking is dat reacties tussen ionen en neutrale moleculen een cruciaal onderdeel zijn van de chemie in het ISM, en dat deze reacties de vorming van complexe moleculen in gang zetten en daardoor de chemie van het ISM beïnvloeden. Dit leverde ook nieuwe inzichten over de oorsprong van het leven in het universum, omdat de ontdekking van deze complexe organische moleculen in de ruimte suggereert dat deze bouwstenen van het leven veel kunnen voorkomen in de kosmos.

Spectroscopische methoden zijn cruciaal bij het bestuderen van interstellaire moleculen en (exo-) planetaire atmosferen. Deze methoden zijn en blijven belangrijk voor het ontdekken van nieuwe interstellaire moleculen en voor het begrijpen van de chemie in de ruimte en in (exo-) planetaire atmosferen. Het introductie hoofdstuk \ref{chapter:intro} geeft een gedetailleerde inleiding over moleculaire ionen in de ruimte, astrochemie en de ontwikkeling van spectroscopische methoden voor het bestuderen van moleculaire ionen in een cryogene ionenval.

Cryogene ionenvallen zijn belangrijke hulpmiddelen in het vakgebied van de spectroscopie, met name voor het bestuderen van moleculaire ionen. Ze stellen ons in staat om actiespectroscopie in de gasfase uit te voeren op koude moleculaire ionen bij een temperatuur van < 10 K, d.w.z. een simulatie van de omstandigheden in het ISM. De combinatie van actiespectroscopische methoden en cryogene ionenvallen biedt verschillende voordelen ten opzichte van traditionele spectroscopiemethoden, waaronder massaselectie en opslag in een koude ionenval, niet-verontreinigde spectra en smalle lijnbreedtes. Hoofdstuk \ref{chapter:methods} geeft een gedetailleerde beschrijving van de instrumentele opstelling en verschillende meettechnieken die in deze studie worden gebruikt voor de karakterisering van de rotationele en vibrationele overgangen van koude moleculaire ionen.

De volgende hoofdstukken \ref{chapter:C3H3+}, \ref{chapter:CH3CNH+} en \ref{chapter:HC3N+} geven een karakterisering van de vibrationele overgangen van potentiële interstellaire moleculaire ionen. Ze worden experimenteel (met behulp van IRPD) en theoretisch onderzocht en de resultaten zijn als volgt:

Het koolwaterstofion C$_3$H$_3^+$ is een belangrijk intermediair in verschillende astronomische omgevingen, zoals het 
interstellaire medium, de coma van kometen en planetaire atmosferen. Daarom karakteriseren we de isomeren van 
[C$_3$H$_3$]$^+$ en [C$_3$D$_3$]$^+$ in hoofdstuk \ref{chapter:C3H3+}. Aangezien deze ionen twee stabiele isomeren hebben, namelijk het 
cyclische cyclopropenyl-kation, c-C$_3$H$_3^+$, en het lineaire propargyl-kation, H$_2$C$_3$H$^+$, worden de isomeren 
gekwantificeerd bij verschillende ionen-broncondities en met verschillende beginstoffen met behulp van kinetische 
scans. Hierbij wordt het verdwijnen van het ioncomplex gemonitord als functie van de tijd en in aanwezigheid van 
resonante straling voor een specifieke vibrationele overgang voor dat isomeer.

Methylcyanide (CH$_3$CN) was een van de eerste moleculen die tijdens radio-astronomische waarnemingen van het ISM werd gedetecteerd. Omdat methylcyanide een hoge protonaffiniteit heeft, veel groter dan die van H$_2$, kan de geïoniseerde versie (CH$_3$CNH$^+$) effectief in het interstellaire medium gevormd worden via exotherme protonoverdracht van H$^+$ naar CH$_3$CN.
Hoofdstuk \ref{chapter:CH3CNH+} richt zich op CH$_3$CNH$^+$ en geeft een uitgebreide experimentele en kwantumchemische studie van het vibrationele spectrum van Ne-CH$_3$CNH$^+$. De invloed van het zwak gebonden neonatoom op de IRPD-experimenten wordt gedetailleerd onderzocht met behulp van verschillende computationele methoden.

Cyanopolyynes zijn een klasse van moleculen bestaande uit een keten van koolstof- en stikstofatomen. Het eenvoudigste cyanopolyyne molecuul is cyanoacetyleen (HC$_3$N), een van de meest voorkomende polyatomische soorten in het ISM. Het is in verschillende astronomische omgevingen waargenomen, waaronder het interstellaire medium, kometen en de atmosfeer van Titan, een van de manen van Saturnus. Het sterk reactieve kation (HC$_3$N$^+$) die mogelijk ontstaat uit de ionisatie van HC$_3$N door kosmische straling of UV-fotonen is echter nog niet in de ruimte gedetecteerd. Daarom onderzoeken we in hoofdstuk \ref{chapter:HC3N+} HC$_3$N$^+$ met vibrationele studies. Interessant is dat het HC$_3$N$^+$ moleculaire ion een lineair molecuul is met een niet-gevulde elektronenschil. Dit leidt tot de aanwezigheid van Renner-Teller-effecten, d.w.z. vibronische koppelingen. Hierdoor wordt de Born-Oppenheimer benadering ontoepasbaar en word het RT-effect geanalyseerd met behulp van een effectieve benadering van de Hamiltoniaan. De invloed van de tag in IRPD, met name op de RT-be\"{i}nvloede buigvibraties, wordt in detail besproken.

De laatste twee hoofdstukken, \ref{chapter:CD+} en \ref{chapter:CO+}, richten zich op het onderzoek van hoge resolutie puur 
rotationele-actiespectroscopie. Rotationele overgangen leveren vergelijkbare of zelfs beter onderscheidende moleculaire 
vingerafdrukken op dan vibrationele overgangen. Vanwege hun lage excitatietemperatuur worden de meeste interstellaire 
moleculen geïdentificeerd aan de hand van hun rotationele overgangen. In deze scriptie gebruiken we de hoge resolutie 
rotationele-actie spectroscopietechniek die ROtational State-dependent Attachment of rare gas Atoms (ROSAA) heet. Deze 
methode gebruikt een verandering in de complexatiesnelheid van edelgasatomen met ionen om pure rotationele overgangen 
van ionen te meten. Het belangrijkste doel van de laatste twee hoofdstukken is om het ROSAA-proces en de 
signaalintensiteiten ervan in detail te begrijpen met de ondersteuning van numerieke simulatieresultaten. Daarom worden 
de implementatie en het onderzoek van deze nieuwe rotationele-actiespectroscopietechniek geïllustreerd voor een ion met 
een gevulde elektronenschil (CD$^+$ in hoofdstuk \ref{chapter:CD+}) en voor de eerste keer, een moleculair ion met een 
niet-gevulde elektronenschil (CO$^+$ in hoofdstuk \ref{chapter:CO+}). In hoofdstuk \ref{chapter:CD+} wordt de kinetiek 
van termoleculaire ion-neutraal botsingsprocessen onderzocht om vervolgens systematisch een ROSAA-kinetisch model te 
ontwikkelen. hoofdstuk \ref{chapter:CO+} geeft een gedetailleerd beeld van het gebruik van dit model voor het 
niet-gevulde-elektronenschil molecuul-ion CO$^+$, dat rotationeel werd gekarakteriseerd met de Zeeman-splitsing als 
gevolg van het magnetisch veld van de aarde.

In de volgende paragrafen wordt een kort vooruitzicht gegeven over mogelijke toekomstige studies.

Zoals hierboven besproken, kan het meten van de rotationele overgangen van moleculaire ionen de mogelijkheid om ze 
astronomisch te detecteren aanzienlijk vergroten. Daarom zullen de moleculaire ionen die in dit proefschrift een 
vibrationele karakterisering hebben gekregen, verder kunnen worden onderzocht met behulp van de ROSAA-methode. De 
vibrationele spectra en kwantumchemische berekeningen die in dit werk zijn geleverd, vormen samen met het ontwikkelde 
numerieke model een solide basis voor deze toekomstige studies. Het propargylkation l-C$_3$H$_3^+$ (d.w.z. CH$_2$CCH$^+$) is een belangrijke kandidaat om te bestuderen, aangezien de neutrale vorm ervan onlangs (2021) is gedetecteerd in TMC-1 en er is aangetoond dat het een van de meest voorkomende radicalen is.

Ook kan het interessant zijn om de rotationele overgangen van de enkelvoudig en dubbel gedeutereerde vormen van c-C$_3$H$_3^+$ (de niet-gedeutereerde cyclische variant c-C$_3$H$_3^+$ heeft geen permanent dipoolmoment) te bestuderen. 

In een ander scenario hebben we geprobeerd de rotationele overgangen van het lineaire isomeer van het [CH$_2$CN]$^+$ ion te meten, d.w.z. het H$_2$CCN$^+$ (cyanomethyl) moleculaire ion, dat recentelijk in onze groep is bestudeerd met behulp van IRPD-spectroscopie. Tot nu toe zijn deze rotationele overgangen niet gedetecteerd met ROSAA, wat verder moet worden onderzocht met behulp van het ontwikkelde ROSAA-model. De zoektocht naar de rotationele overgangen was gebaseerd op hoogwaardige kwantummechanische berekeningen, wat het zoekgebied beperkte tot 1 GHz, waarvoor eerder gemeten IRPD-gegevens als een goede benchmark fungeerden. In toekomstige studies kan het model dat in deze scriptie beschreven is gemakkelijk worden gebruikt om de verwachte signaalintensiteit bij een bepaalde temperatuur, deeltjesdichtheid en lichtsterkte te begrijpen en de toepasbaarheid van de ROSAA-techniek te evalueren.

De belangrijkste voordelen van de ROSAA-techniek zijn (1) de mogelijkheid van heliumbinding aan bijna elk kation bij lage temperaturen ($<15$ K) waardoor het een zeer algemeen toepasbare techniek is die het bestuderen van lastige en zeer reactieve moleculaire ionen mogelijk maakt en (2) dat de techniek zeer gevoelig is in vergelijking met traditionele absorptiemethoden, d.w.z. dat slechts enkele honderden ionen nodig zijn om een spectroscopisch signaal te kunnen detecteren. Het vereist echter over het algemeen een lange integratietijd om een hoge signaal-ruisverhouding te verkrijgen. Het duurt ongeveer ~28 uur om 50 MHz te scannen in stappen van 10 kHz gedurende 10 iteratiecycli, wat resulteert in een signaal-ruisverhouding van $3\%$, waarmee ROSAA-signalen met ongeveer drie keer deze waarde kunnen worden gedetecteerd. Dit is een essentiële stap omdat de ROSAA spectroscopische signaalintensiteit rechtstreeks verband houdt met de verandering in het aantal ioncomplexen dat wordt gevormd als gevolg van een verandering in de interne populatieverdeling van de rotationele niveaus op de resonantiefrequentie. Zwakke signalen worden ook verwacht wanneer er slechts een paar ionen aanwezig zijn in een rotationeel niveau bij een bepaalde temperatuur, waardoor een zeer kleine verandering in de populatie kan worden veroorzaakt bij resonantie. Het is echter inderdaad mogelijk om dergelijke kleine veranderingen te detecteren door een langere integratietijd zoals eerder besproken, zelfs voor ionen die zeer lastig te produceren en te bestuderen zijn.

Daarom kan men in de toekomst de ROSAA-techniek gebruiken en nauwkeurige rotationele constanten bepalen van bijvoorbeeld carbokationen en stikstof-bevattende koolstofketen-radicaalionen. Deze informatie kan worden gebruikt voor de astronomische identificatie van de eerder genoemde radicaalionen, die belangrijke intermediairen zijn voor veel organische moleculen die overvloedig worden waargenomen in het ISM.

}